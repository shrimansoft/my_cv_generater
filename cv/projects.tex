\cvsection{Projects}

\begin{cvpubs}

  \cvpub{
    \textbf{AMaR :} Autonomous Multi-utility Rover is an affordable robotic
    system that can substitute human interference in potentially hazardous
    scenarios.}

\begin{tabularx}{\textwidth}{ X  X }
	  \textbf{Contribution:} Responsible for the Software part.  &  \textbf{Link:} https://www.niser.ac.in/~smishra/club/rtc/amar \\
\end{tabularx}


  \cvpub{
    \textbf{NIRMAL :} WiFi-enabled smart hands-free sanitiser dispenser
    system. Sends email alert to the operator when sanitiser level in container
    drops below a threshold. Powered by NodeMCU.}

\begin{tabularx}{\textwidth}{ X  X }
	  \textbf{Contribution:} Wrote
  the server code that connects to sanitisers and it generates a notification
  email when it needs to be refiled. &
	    \textbf{Link:} http://niser.ac.in/~smishra/project/nirmal \\
\end{tabularx}



  \cvpub{\textbf{Scorpion :} A small
    Autonomous Multiutility Rover type bot designed to adapt to a wide array of
    use cases. The system is Powered by Nvidea Jetson Nano and Lidar Sensor.}


\begin{tabularx}{\textwidth}{ X  X }
	  \textbf{Contribution:} Write an API for the rover that can be accessed through
  a remote computer. &
	    \textbf{Link:} https://www.niser.ac.in/~smishra/club/rtc/scorpion \\
\end{tabularx}



  \cvpub{
    \textbf{OMR reader :}This program reads an image of OMR sheets and
    generates the data of its entry. It is used to check the copies of the exam
    conducted by For Zariya https://github.com/shrimansoft/sciquest}

\begin{tabularx}{\textwidth}{ X  X }
	  \textbf{Contribution:} solo project.  &
	    \textbf{Link:} https://github.com/shrimansoft/sciquest\\
\end{tabularx}

  \cvpub{\textbf{Exceptional point : } Under the guidance of Dr. Kush Saha in
    Spring of 2021 }

  \textbf{Abstract : } We studied the mathematical formulation of exceptional
  points in a general $2 \times 2$ non-Hermitian Hamiltonian that is a function
  of a complex parameter.

  \cvpub{\textbf{Dynamics of Newton's and Halley's Method :}
    Under the guidance of
    Dr Ritwik Mukherjee, Dr Sayantani Bhattacharyya in Fall of 2022 }

  \textbf{Abstract : }
  We studied the Basins of convergence by Newton's and Halley’s Methods on the complex quadratic polynomial.
  We found that the iteration converges to the root closer to the initial choice in both methods.

  \cvpub{\textbf{Convergence analysis of Newton's and Secant
      methods and their generalisations :}
    Under the guidance of
    Dr Ritwik Mukherjee, Dr Sayantani Bhattacharyya in Spring of 2023 }

  \textbf{Abstract : }
  We show that the order of convergence of the secant method is the golden
  ratio(This is not new work). We introduced a Variant of Newton's method where instead of
  fitting the straight line, we fit the parabola. We analysed its convergence and its order
  of convergence.


\end{cvpubs}
